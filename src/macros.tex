% 255 = 1, ..., 0 = 0
\definecolor{lightgray}{rgb}{0.9725,0.9725,0.9725}

\DeclareMathOperator\rank{rank}
\DeclareMathOperator\Span{Span}
\DeclareMathOperator\Col{Col}
\DeclareMathOperator\Row{Row}
\DeclareMathOperator\Null{Null}
\DeclareMathOperator\adj{adj}
\DeclareMathOperator\trace{tr}
\DeclareMathOperator\diag{diag}
\DeclareMathOperator\Hom{Hom}
\DeclareMathOperator\Floor{floor}
\DeclareMathOperator\Ceil{ceil}

\DeclarePairedDelimiter{\abs}{\lvert}{\rvert}
\DeclarePairedDelimiter{\norm}{\lVert}{\rVert}
\DeclarePairedDelimiter{\ceil}{\lceil}{\rceil}
\DeclarePairedDelimiter{\floor}{\lfloor}{\rfloor}

\newcommand*{\pref}[2]{(#1 \ref{#2})}
\newcommand*{\defines}{\coloneqq}
\newcommand*{\bolddot}{\boldsymbol{\cdot}}
\newcommand*{\setbuild}{\:\middle|\:}
\newcommand*{\vertbar}{\rule[-1ex]{0.5pt}{2.5ex}}
\newcommand*{\horzbar}{\rule[.5ex]{2.5ex}{0.5pt}}
\newcommand*{\proofright}{"$\Rightarrow$" }
\newcommand*{\proofleft}{"$\Leftarrow$" }
\newcommand*{\notimplies}{\mathrel{{\ooalign{\hidewidth$\not\phantom{=}$\hidewidth\cr$\implies$}}}}
\newcommand*{\tolim}[2]{\underset{#1\to#2}{\xrightarrow{\hspace*{1.1cm}}}}
\newcommand*{\seqinfty}[1]{\tolim{#1}{\infty}}
\newcommand*{\domain}[1]{\mathcal{D}(#1)}
\newcommand*{\codomain}[1]{\mathcal{C}(#1)}
\newcommand*{\image}[1]{\text{Im}(#1)}
\newcommand*{\kernel}[1]{\text{Ker}(#1)}
\newcommand*{\grad}[1]{\text{grad}\,#1}
\newcommand*{\divergence}[1]{\text{div}\,#1}
\newcommand*{\curl}[1]{\text{curl}\,#1}
\newcommand*\diff{\mathop{}\!\mathrm{d}}
\newcommand*{\evalat}[3]{\left.#1\vphantom{\int}\right\rvert_{#2}^{#3}}
\newcommand*{\hati}{\hat{\imath}}
\newcommand*{\hatj}{\hat{\jmath}}
\newcommand*{\hatk}{\hat{k}}
\newcommand*{\inlinematrix}[1]{\left(\begin{smallmatrix}#1\end{smallmatrix}\right)}

\renewcommand{\Re}{\operatorname{\mathfrak{Re}}}
\renewcommand{\Im}{\operatorname{\mathfrak{Im}}}

\makeatletter
\renewcommand*\env@matrix[1][*\c@MaxMatrixCols c]{%
  \hskip -\arraycolsep
  \let\@ifnextchar\new@ifnextchar
  \array{#1}}
\makeatother
